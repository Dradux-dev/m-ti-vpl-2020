\begin{coverpages}
	
	\setlength{\parskip}{10pt}
	\setlength{\parindent}{0pt}
	
	\vspace{2cm}
	\normalsize
	\begin{tabularx}{\textwidth}{ p{13.75cm} | r }
	    \toprule
		\textbf{\thislecture}  & \multirow{4}{*}{\quad 
		\includegraphics[width=2.5cm]{BilderPDF/beuth.eps}} \\
		\thisexam, \thisdate, \thissemester &  \\
		\cline{1-1}
		\emph{Thema:}  & \\
		\thistheme &  \\
		\bottomrule
	\end{tabularx}

	
	\vspace{2cm}
	\LARGE\textbf{Hinweis zur Durchführung der Laborversuche/Hausarbeiten}
	
\normalsize
Die Skripte bzw. Literaturhinweise zu den Laborversuchen/Hausarbeiten werden in der Vorlesung bzw. online (Beuth Moodle) zur
Verfügung gestellt. Es sollten ausschlie\ss lich aktuelle Skripte verwendet werden. Spezielle Hinweise
zu einem Laborversuch/Hausarbeit sind am Anfang des entsprechenden Laborskripts zu finden.
	
\textbf{Ablauf}

\begin{itemize}
\item Die Laborversuche/Hausarbeiten sind derart konzeptioniert, dass diese innerhalb eines Labortermins bzw. einer definierten Abgabefrist, falls nicht anders angegeben, zu bearbeiten sind.
\item Ein Laborversuch/Hausarbeit besteht aus Selbststudium, Versuchsvorbereitung, Versuchsdurchführung und einem Protokoll. Zur Versuchsvorbereitung gehört eine analytische Herleitung der angestrebten Realisierung bzw. Umsetzung. Diese sind in der Ausarbeitung vollständig zu dokumentieren. %Diese ist vor Versuchsbeginn dem begleitenden Laboringenieur und/oder dem Dozenten unaufgefordert vorzulegen. 
\item Projekte/Laborversuche: Es sollen selbstständig Gruppen mit jeweils 2 Studierenden gebildet werden. 
\item Hausarbeiten: Sind selbstständig zu bearbeiten. Lerngruppen sind empfohlen. 
\item Das Laborprotokoll ist mit unterschriebenem Deckblatt 4 Wochen nach dem Versuch abzugeben. Es wird Ihnen das originale Latex-Dokument zur Verfügung gestellt. Dieses können Sie dann mit ihren Ausarbeitungen vervollständigen. Bei handschriftlichen Ausführungen versehen Sie bitte jedes Blatt mit Namen und Matrikelnummer und nummerieren Sie alle Blätter durch.
\item Bei Quelltexten wie z.B. Python, Matlab-Code usw. ist auf eine erklärende Kommentierung zu achten. Ein Ablaufdiagramm oder andere graphische Methoden zur Erläuterung des eigenen Lösungsansatzes sind zu ergänzen.
%\item Hinweise in Vorlesungen/Übungen sowie durch die Betreuer sind zu berücksichtigen.
\end{itemize}

Es sind alle Arbeits- und Ableitungsschritte zu dokumentieren. Lösungen ohne erkennbaren Lösungsweg oder kurzer Begründung, sowie durchgestrichene oder nicht lesbare Lösungen werden nicht gewertet. Bei zwei angegebenen Lösungen wird keine berücksichtigt. Bitte benutzen Sie keine roten oder grünen Stifte.

	\vspace{2cm}
	\hrule
	\emph{Ab hier bitte keine Eintragungen vornehmen!}
	
	\vspace{2cm}

\begin{tabularx}{\textwidth}{ p{6cm} | p{6cm} | p{4cm} }
	    \toprule
         Gruppenname: &  & \\
         \cline{1-2}
         Koopke, Philipp & \todo{Matrikelnummer} & Testat \\
         \hline
         Haß, Christian & 903643  & Datum: \\
         \hline
         Schwarzinger, Tarek & \todo{Matrikelnummer} &  Unterschrift:\\
		\bottomrule
	\end{tabularx}
	
\end{coverpages}