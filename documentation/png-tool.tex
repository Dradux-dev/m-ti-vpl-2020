
%%%%%%%%%%%%%%%%%%%%%%% PNG-Tool %%%%%%%%%%%%%%%%%%%%%%%%%%%%%%%%%%%%%%%
\section{PNG-Tool}

%%%%%%%%%%%%%%%%%%%%%%% Unterpunkt - Idee %%%%%%%%%%%%%%%%%%%%%%%%%%%%%%
\subsection{Idee}

Im Bereich des Maschinellen Lernens werden sog. Trainingsdaten benötigt. Diese Daten sollen dem neuronalen Netz als Lernhilfe dienen. Denn ohne ein ausreichendes Training des neuronalen Netzes kann es keine hinreichend präzisen Entscheidungen über bestimmte Merkmale treffen.
Grundsätzlich gilt, je mehr Trainingsdaten verwendet werden, desto präziser wahrscheinlicher ist ein korrektes Ergebnis. Am Ausgang des neuronalen Netzes. \\

Das konkrete Ziel in dieser Arbeit ist die Erkennung von speziellen Flächen anhand ihrer Form. Um nun beliebige Formen zufällig im Raum zu platzieren und diese wahlweise einzufärben wurde dieses PNG-Tool entwickelt. Der Nutzer hat die Möglichkeit über die zur Verfügung stehenden Startparameter alle Einstellungen vorzunehmen. Auf diesem Weg können schnell und ohne weiteres Eingreifen beliebig viele Trainingsbilder erzeugt werden. \\

Zusätzlich wird für jedes Trainingsbild eine individuelle .json Datei angelegt. Diese Datei beinhaltet konkrete Informationen zu denen im Bild befindlichen Objekten. Dies ist wichtig, da das neuronale Netz am Ende mit einer Referenz Vergleich muss ob die Objekte auch korrekt erkannt wurden. Somit tritt ein Lerneffekt ein.  
\newpage

%%%%%%%%%%%%%%%%%%%%%%% Unterpunkt - Umsetzung %%%%%%%%%%%%%%%%%%%%%%%%%
\subsection{Umsetzung}



\newpage

%%%%%%%%%%%%%%%%%%%%%%% Unterpunkt - How to use? %%%%%%%%%%%%%%%%%%%%%%%
\subsection{Bedienung}

Das Programm kann vollständig über die Kommandozeilenparameter konfiguriert werden. 

\newpage

%%%%%%%%%%%%%%%%%%%%%%% Unterpunkt - Anpassbarkeit Formen %%%%%%%%%%%%%%
\subsection{Anpassbarkeit der Formen}

Innerhalb des Projektordners gibt es einen Unterordner "`forms"'. Um eine neue Form hinzuzufügen ist es zunächst notwendig eine neue Datei innerhalb dieses Ordners anzulegen. Der Dateiname muss dabei der beinhaltenden Klasse der Form entsprechen. Jede Form besitzt einen Konstruktor und eine generate() Funktion. Je nach Art der Implementierung gibt es zwei verschiedene Wege eine Form anzulegen. Diese werden im folgenden genauer erläutert. \\

%%%%%%%%%%%%%%%%%%%%%%%%%%%%%%%%%%%
%% EIN POLYGON %%%%%%%%%%%%%%%%%%%%
%%%%%%%%%%%%%%%%%%%%%%%%%%%%%%%%%%%

\subsubsection{Für Formen mittels Polygon-Darstellung}

Es gilt dabei zu beachten, dass alle Formen, welche auf Basis von Polygonen erzeugt werden, von der Klasse PolygonForm erben. Ausgenommen von dieser Regel ist beispielsweise die Form Kreis. Innerhalb des Konstruktors der Form ist im Fall von Polygondarstellung auch der Konstruktor der Basisklasse aufzurufen. Wobei dieser die Größe des Objektes bekommt.\\

Die generate() Funktion dient dem Erzeugen des Objektes. Innerhalb der Funktion werden die Eckpunkte des zu erzeugenden Objektes relativ angegeben. Relativ bedeutet, dass die Koordinaten für X und Y sich in einem Bereich von 0 - 1 bewegen. Ein Aufruf von generatePolygon() mit den Koordinaten als Parameter skaliert dann das Objekt in Abhängigkeit von der zu erzeugenden Größe, welche im Konstruktor übergeben wurde. Dabei handelt es sich um Funktionalität der Basisklasse, welche nicht neu implementiert werden muss. \\

Nachfolgenden ist die beispielhafte Implementierung einer solchen Klasse anhand des Beispieles \textit{Dreieck} zu sehen:\\

\lstinputlisting[language=Python]{../source/tools/png-generator/forms/Triangle.py}

\newpage

%%%%%%%%%%%%%%%%%%%%%%%%%%%%%%%%%%%
%% MEHRERE POLYGONE %%%%%%%%%%%%%%%
%%%%%%%%%%%%%%%%%%%%%%%%%%%%%%%%%%%

\subsubsection{Erstellung einer Form auf Grundlage mehrerer Polygone}

\newpage

%%%%%%%%%%%%%%%%%%%%%%%%%%%%%%%%%%%
%% KEIN POLYGON %%%%%%%%%%%%%%%%%%%
%%%%%%%%%%%%%%%%%%%%%%%%%%%%%%%%%%%

\subsubsection{Für Formen ohne Polygon-Darstellung}

Erfolgt die Implementierung einer neuen Form nicht auf Grundlage von Polygonen, so ist eine eigene Implementierung der Funktion generate() notwendig. Ein Beispiel für einen solchen Fall stellt die Klasse "`Kreis"' dar. Die Kreisberechnung erfolgt auf der Grundlage von Abstandberechnungen zum Mittelpunkt mittels Pythagoras. Ist der Abstand eines Pixels auf diesem Weg größer als der Radius des Kreises, befindet sich das Pixel nicht im Kreis. Es folgt die Implementierung der Klasse "`Kreis"': \\


\lstinputlisting[language=Python]{../source/tools/png-generator/forms/Circle.py}

\newpage

%%%%%%%%%%%%%%%%%%%%%%% Unterpunkt - Anpassbarkeit Farbgenerierung %%%%%
\subsection{Anpassbarkeit der Farbgenerierung}

\anno{Dieser Abschnitt wird in einer späteren Fassung ergänzt.}

\newpage